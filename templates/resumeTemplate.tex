%%%%%%%%%%%%%%%%%%%%%%%%%%%%%%%%%%%%%%%%%
% Medium Length Professional CV
% LaTeX Template
% Version 2.0 (8/5/13)
%
% This template has been downloaded from:
% http://www.LaTeXTemplates.com
%
% Original author:
% Trey Hunner (http://www.treyhunner.com/)
%
% Important note:
% This template requires the resume.cls file to be in the same directory as the
% .tex file. The resume.cls file provides the resume style used for structuring the
% document.
%
%%%%%%%%%%%%%%%%%%%%%%%%%%%%%%%%%%%%%%%%%

%----------------------------------------------------------------------------------------
%	PACKAGES AND OTHER DOCUMENT CONFIGURATIONS
%----------------------------------------------------------------------------------------

\documentclass{resume} % Use the custom resume.cls style

\usepackage[left=0.75in,top=0.6in,right=0.75in,bottom=0.6in]{geometry} % Document margins

\name{<$ basics.name $>} % Your name
\address{<$ basics.location.city $>, <$ basics.location.region $> <$ basics.location.countryCode $>} % Your address
%\address{123 Pleasant Lane \\ City, State 12345} % Your secondary addess (optional)
%\address{(000)~$\cdot$~111~$\cdot$~1111 \\ john@smith.com} % Your phone number and email
\address{<$ basics.email $>}
\address{<$ basics.website $>}

\begin{document}



%----------------------------------------------------------------------------------------
%	WORK EXPERIENCE SECTION
%----------------------------------------------------------------------------------------

\begin{rSection}{Experience}

<% for job in work %>

\begin{rSubsection}{<$ job.company $>}{<$ job.startDate $> - <$ job.endDate $>}{ <$ job.position $> }{} 
{ <$ job.summary | TeXEscape $> }
<% for highlight in job.highlights %>
\item <$ highlight | TeXEscape $>
<% endfor %>
\end{rSubsection}

<% endfor %>

\end{rSection}

%----------------------------------------------------------------------------------------
%	EXAMPLE SECTION
%----------------------------------------------------------------------------------------

\begin{rSection}{Awards}

<% for award in awards %>

{\bf <$ award.awarder $> --- <$ award.title $>} \hfill {\em <$ award.date $>} \\ 
<$ award.summary $>\\

<% endfor %>


\end{rSection}

%----------------------------------------------------------------------------------------

%----------------------------------------------------------------------------------------
%	TECHNICAL STRENGTHS SECTION
%----------------------------------------------------------------------------------------

\begin{rSection}{Technical Strengths}

\begin{tabular}{ @{} >{\bfseries}l @{\hspace{6ex}} l }
Computer Languages & Prolog, Haskell, AWK, Erlang, Scheme, ML \\
Protocols \& APIs & XML, JSON, SOAP, REST \\
Databases & MySQL, PostgreSQL, Microsoft SQL \\
Tools & SVN, Vim, Emacs
\end{tabular}

\end{rSection}

%----------------------------------------------------------------------------------------
%	EDUCATION SECTION
%----------------------------------------------------------------------------------------

\begin{rSection}{Education}

<% for school in education %>

{\bf <$ school.institution $>} \hfill {\em <$ school.startDate $> --- <$ school.endDate $>} \\ 
<$ school.studyType $> in <$ school.area $>\\
Overall GPA: <$ school.gpa $>

<% endfor %>

\end{rSection}

%----------------------------------------------------------------------------------------
%	LANGUAGES SECTION
%----------------------------------------------------------------------------------------

\begin{rSection}{Languages}

\begin{tabular}{ @{} >{\bfseries}l @{\hspace{6ex}} l }
<% for lang  in languages %>
<$ lang.language $> & <$ lang.fluency $> \\
<% endfor %>
\end{tabular}


\end{rSection}


%----------------------------------------------------------------------------------------
%	INTERESTS SECTION
%----------------------------------------------------------------------------------------

\begin{rSection}{Interests}

\begin{tabular}{ @{} >{\bfseries}l @{\hspace{6ex}} l }
<% for interest  in interests %>
<$ interest.name $> & 
<%- set comma = joiner() -%> 
<%- for keyw in interest.keywords -%> 
<$ comma() $> <$ keyw $> 
<%- endfor -%>\\
<% endfor %>
\end{tabular}

\end{rSection}

%----------------------------------------------------------------------------------------
%	EXAMPLE SECTION
%----------------------------------------------------------------------------------------

%\begin{rSection}{Section Name}

%Section content\ldots

%\end{rSection}

%----------------------------------------------------------------------------------------

\end{document}
